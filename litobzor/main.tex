\documentclass[a4paper,14pt]{article}

%%% Работа с русским языком
\usepackage{cmap}					% поиск в PDF
\usepackage{mathtext} 				% русские буквы в формулах
\usepackage[T2A]{fontenc}			% кодировка
\usepackage[utf8]{inputenc}			% кодировка исходного текста
\usepackage[english,russian]{babel}	% локализация и переносы
\usepackage{indentfirst}
\frenchspacing

\renewcommand{\epsilon}{\ensuremath{\varepsilon}}
\renewcommand{\phi}{\ensuremath{\varphi}}
\renewcommand{\kappa}{\ensuremath{\varkappa}}
\renewcommand{\le}{\ensuremath{\leqslant}}
\renewcommand{\leq}{\ensuremath{\leqslant}}
\renewcommand{\ge}{\ensuremath{\geqslant}}
\renewcommand{\geq}{\ensuremath{\geqslant}}
\renewcommand{\emptyset}{\varnothing}

%%% Дополнительная работа с математикой
\usepackage{amsmath,amsfonts,amssymb,amsthm,mathtools} % AMS
\usepackage{icomma} % "Умная" запятая: $0,2$ --- число, $0, 2$ --- перечисление

%% Номера формул
%\mathtoolsset{showonlyrefs=true} % Показывать номера только у тех формул, на которые есть \eqref{} в тексте.
%\usepackage{leqno} % Нумереация формул слева

%% Свои команды
\DeclareMathOperator{\sgn}{\mathop{sgn}}

%% Перенос знаков в формулах (по Львовскому)
\newcommand*{\hm}[1]{#1\nobreak\discretionary{}
{\hbox{$\mathsurround=0pt #1$}}{}}

%%% Работа с картинками
\usepackage{graphicx}  % Для вставки рисунков
\graphicspath{{images/}{images2/}}  % папки с картинками
\setlength\fboxsep{3pt} % Отступ рамки \fbox{} от рисунка
\setlength\fboxrule{1pt} % Толщина линий рамки \fbox{}
\usepackage{wrapfig} % Обтекание рисунков текстом

%%% Работа с таблицами
\usepackage{array,tabularx,tabulary,booktabs} % Дополнительная работа с таблицами
\usepackage{longtable}  % Длинные таблицы
\usepackage{multirow} % Слияние строк в таблице

%%% Теоремы
\theoremstyle{plain} % Это стиль по умолчанию, его можно не переопределять.
\newtheorem{theorem}{Теорема}[section]
\newtheorem{proposition}[theorem]{Утверждение}
 
\theoremstyle{definition} % "Определение"
\newtheorem{corollary}{Следствие}[theorem]
\newtheorem{problem}{Задача}[section]
 
\theoremstyle{remark} % "Примечание"
\newtheorem*{nonum}{Решение}

%%% Программирование
\usepackage{etoolbox} % логические операторы

%%% Страница
\usepackage{extsizes} % Возможность сделать 14-й шрифт
\usepackage{geometry} % Простой способ задавать поля
	\geometry{top=20mm}
	\geometry{bottom=20mm}
	\geometry{left=30mm}
	\geometry{right=15mm}
 %
%\usepackage{fancyhdr} % Колонтитулы
% 	\pagestyle{fancy}
 	%\renewcommand{\headrulewidth}{0pt}  % Толщина линейки, отчеркивающей верхний колонтитул
% 	\lfoot{Нижний левый}
% 	\rfoot{Нижний правый}
% 	\rhead{Верхний правый}
% 	\chead{Верхний в центре}
% 	\lhead{Верхний левый}
%	\cfoot{Нижний в центре} % По умолчанию здесь номер страницы

\linespread{1.3} % полуторный интервал
\frenchspacing

\usepackage{lastpage} % Узнать, сколько всего страниц в документе.

\usepackage{soul} % Модификаторы начертания

\usepackage{hyperref}
\usepackage[usenames,dvipsnames,svgnames,table,rgb]{xcolor}
\hypersetup{				% Гиперссылки
	unicode=true,           % русские буквы в раздела PDF
	pdftitle={Заголовок},   % Заголовок
	pdfauthor={Автор},      % Автор
	pdfsubject={Тема},      % Тема
	pdfcreator={Создатель}, % Создатель
	pdfproducer={Производитель}, % Производитель
	pdfkeywords={keyword1} {key2} {key3}, % Ключевые слова
	colorlinks=true,       	% false: ссылки в рамках; true: цветные ссылки
	linkcolor=black,          % внутренние ссылки
	citecolor=black,        % на библиографию
	filecolor=balck,      % на файлы
	urlcolor=black           % на URL
}


\usepackage{csquotes} % Еще инструменты для ссылок

\usepackage[backend=biber,bibencoding=utf8,sorting=nty,maxcitenames=2,style=numeric-comp]{biblatex}
\addbibresource{bib1.bib}

\usepackage{multicol} % Несколько колонок

\usepackage{tikz} % Работа с графикой
\usepackage{pgfplots}
\usepackage{pgfplotstable}

\usepackage[nottoc,numbib]{tocbibind}

\usepackage{titlesec}

\usepackage[titletoc]{appendix}



\begin{document} 
	
\section{Обзор литературы}	
Одним из основных недостатков водород-воздушных топливных элементов на протон-обменных мембранах, который ограничивает их массовое внедрение является их малое время работы до необходимого обслуживания и ремонта \cite{Garland2007}. 
Именно поэтому ресурс топливных элементов является одной из основных тем исследований \cite{Yonoff2019}, причём доля статей, посвящённых этой теме, с каждым годом только растёт, что видно на рисунке \ref{fig:research_trends}.  

\begin{figure} [h]
	\centering
	\includegraphics[width=0.7\linewidth]{images/research_trends}
	\caption[Распределение статей по различным темам]{На графике обозначено распределение по годам доли статей по темам от всех статей по теме ПОМТЭ}
	\label{fig:research_trends}
\end{figure}

Любой блок топливных элементов состоит из множества комплектующих, каждый из которых имеет свой ресурс. 
При этом ограниченное время работы всего устройства обычно связано со снижением характеристик либо мембраны, либо каталитического слоя \cite{Jahnke2016}.

Процессы, которые приводят к нарушению функций мембраны можно разделить на две крупные категории \cite{Collier2006}.
Первая категория --- это механическая и термическая деградация. К механической относят возникновение надрывов, трещин и отверстий, а к термической --- необратимое пересыхание и термолиз.
Ко второй категории можно отнести химические и электрохимические механизмы.
Наиболее важный процесс этой категории - это образование гидропероксильного или какого-либо другого свободного радикала, а также дальнейшее разложение мембраны при радикальной атаке.
Важно отметить, что процессы двух категорий плотно связаны друг с другом.
Так, например, химическое разложение мембраны может привести к образованию сквозных отверстий в ней, рост которых будет проходить вследствие механизмов первой категории.

В состав каталитического слоя, помимо других элементов, входит иономер, который по своим способам деградация во многом похожа на мембрану \cite{Okonkwo2021_I}. 
Однако катализатор подвержен и множеству других процессов, снижающих его работоспособность. 
Среди основных механизмов деградации можно назвать отравление, вымывание, оствальдовское созревание и миграцию частиц платины\cite{Okonkwo2021_Pt}, однако помимо них важны и многие другие процессы \cite{Nguyen2021}. 
Важно отметить, что отдельные аспекты этих механизмов неплохо изучены  \cite{Garcia2008, Prokop2020}, однако причину снижения характеристик топливного элемента в каждом конкретном случае удаётся понять далеко не всегда.

Существует множество экспериментальных методов, позволяющих отслеживать изменения в мембранно-электродных блоках, происходящих вследствие деградации. 
Основной темой этой работы является использование дифракции рентгеновского синхротронного излучения, так как такой подход обладает целым рядом преимуществ. 
Данные о широкоугловом рентгеновском рассеянии позволяют с хорошей точностью смоделировать структуру платины в каталитическом слое, описав размер и кристалличность её частиц \cite{Sasaki2016}.
Малоугловое рассеяние может использоваться для получения информации об иономере и макроструктуре каталитического слоя, а так же о характеристиках мембраны \cite{Schmidt2008}.
Микро- и нано-фокусная дифракция позволяют исследовать распределение, например, размера частиц платины в каталитическом слое от расстояния от мембраны, получая более полную информацию о состоянии мембранно-электродного блока. 
Благодаря высокой яркости синхротронного источника есть возможность изучения топливного элемента во время работы при использовании специальной ячейки \cite{Martens2019_cell}, что недоступно для абсолютного большинства других методов изучения.

На данный момент значительная доля исследований мембранно-электродных блоков в рентгеновском диапазоне посвящены изучению распределения и транспорта воды, причём для исследований используются как лабораторные рентгеновские микроскопы \cite{Lee2013, Eller2014}, так и синхротроны \cite{Sasabe2011, Deevanhxay2011}. 
Другая часть таких работ посвящена различным аспектам одного из важнейших процессов, обеспечивающих работу топливного элемента - окислению наночастиц платины. 
Темами различных исследований становились этапы окисления частиц на углеродной подложке \cite{Imai2009}, в различных типах электрохимических ячеек \cite{Sasaki2016} и в составе МЭБ-ов \cite{Martens2019_MEA}.
Немаловажно выделить исследование неоднородности этого процесса по поверхности каталитического слоя и, как следствие, различной скорости деградации топливного элемента в разных его частях \cite{Martens2021}.

Гораздо большую популярность синхротроны снискали с другими топливными элементами --- твёрдооксидными.
Это легко объяснить гораздо более выраженной кристаллической структурой, изменения которой легко заметить с помощью рентгеновской дифракции.
Это позволяет отслеживать напряжения кристаллической решетки \cite{Villanova2010}, определять микроструктуру \cite{Shearing2011} и даже строить трёхмерные модели этого типа топливных элементов \cite{Izzo2008}. 
Также можно отметить некоторые работы, посвящённые изменениям в структуре ТОТЭ при длительной работе \cite{Villanova2019}.


\printbibliography[heading=bibintoc,title={Список литературы}]

\end{document}