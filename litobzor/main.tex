\documentclass[a4paper, 14pt]{article}

%%% Работа с русским языком
\usepackage{cmap}					% поиск в PDF
\usepackage{mathtext} 				% русские буквы в формулах
\usepackage[T2A]{fontenc}			% кодировка
\usepackage[utf8]{inputenc}			% кодировка исходного текста
\usepackage[english,russian]{babel}	% локализация и переносы
\usepackage{indentfirst}
\frenchspacing

\renewcommand{\epsilon}{\ensuremath{\varepsilon}}
\renewcommand{\phi}{\ensuremath{\varphi}}
\renewcommand{\kappa}{\ensuremath{\varkappa}}
\renewcommand{\le}{\ensuremath{\leqslant}}
\renewcommand{\leq}{\ensuremath{\leqslant}}
\renewcommand{\ge}{\ensuremath{\geqslant}}
\renewcommand{\geq}{\ensuremath{\geqslant}}
\renewcommand{\emptyset}{\varnothing}

%%% Дополнительная работа с математикой
\usepackage{amsmath,amsfonts,amssymb,amsthm,mathtools} % AMS
\usepackage{icomma} % "Умная" запятая: $0,2$ --- число, $0, 2$ --- перечисление

%% Номера формул
%\mathtoolsset{showonlyrefs=true} % Показывать номера только у тех формул, на которые есть \eqref{} в тексте.
%\usepackage{leqno} % Нумереация формул слева

%% Свои команды
\DeclareMathOperator{\sgn}{\mathop{sgn}}

%% Перенос знаков в формулах (по Львовскому)
\newcommand*{\hm}[1]{#1\nobreak\discretionary{}
{\hbox{$\mathsurround=0pt #1$}}{}}

%%% Работа с картинками
\usepackage{graphicx}  % Для вставки рисунков
\graphicspath{{images/}{images2/}}  % папки с картинками
\setlength\fboxsep{3pt} % Отступ рамки \fbox{} от рисунка
\setlength\fboxrule{1pt} % Толщина линий рамки \fbox{}
\usepackage{wrapfig} % Обтекание рисунков текстом

%%% Работа с таблицами
\usepackage{array,tabularx,tabulary,booktabs} % Дополнительная работа с таблицами
\usepackage{longtable}  % Длинные таблицы
\usepackage{multirow} % Слияние строк в таблице

%%% Теоремы
\theoremstyle{plain} % Это стиль по умолчанию, его можно не переопределять.
\newtheorem{theorem}{Теорема}[section]
\newtheorem{proposition}[theorem]{Утверждение}
 
\theoremstyle{definition} % "Определение"
\newtheorem{corollary}{Следствие}[theorem]
\newtheorem{problem}{Задача}[section]
 
\theoremstyle{remark} % "Примечание"
\newtheorem*{nonum}{Решение}

%%% Программирование
\usepackage{etoolbox} % логические операторы

%%% Страница
\usepackage{extsizes} % Возможность сделать 14-й шрифт
\usepackage{geometry} % Простой способ задавать поля
	\geometry{top=20mm}
	\geometry{bottom=20mm}
	\geometry{left=30mm}
	\geometry{right=15mm}
 %
%\usepackage{fancyhdr} % Колонтитулы
% 	\pagestyle{fancy}
 	%\renewcommand{\headrulewidth}{0pt}  % Толщина линейки, отчеркивающей верхний колонтитул
% 	\lfoot{Нижний левый}
% 	\rfoot{Нижний правый}
% 	\rhead{Верхний правый}
% 	\chead{Верхний в центре}
% 	\lhead{Верхний левый}
%	\cfoot{Нижний в центре} % По умолчанию здесь номер страницы

\linespread{1.3} % полуторный интервал
\frenchspacing

\usepackage{lastpage} % Узнать, сколько всего страниц в документе.

\usepackage{soul} % Модификаторы начертания

\usepackage{hyperref}
\usepackage[usenames,dvipsnames,svgnames,table,rgb]{xcolor}
\hypersetup{				% Гиперссылки
	unicode=true,           % русские буквы в раздела PDF
	pdftitle={Заголовок},   % Заголовок
	pdfauthor={Автор},      % Автор
	pdfsubject={Тема},      % Тема
	pdfcreator={Создатель}, % Создатель
	pdfproducer={Производитель}, % Производитель
	pdfkeywords={keyword1} {key2} {key3}, % Ключевые слова
	colorlinks=true,       	% false: ссылки в рамках; true: цветные ссылки
	linkcolor=black,          % внутренние ссылки
	citecolor=black,        % на библиографию
	filecolor=balck,      % на файлы
	urlcolor=black           % на URL
}


\usepackage{csquotes} % Еще инструменты для ссылок

\usepackage[backend=biber,bibencoding=utf8,sorting=nty,maxcitenames=2,style=numeric-comp]{biblatex}
\addbibresource{bib1.bib}

\usepackage{multicol} % Несколько колонок

\usepackage{tikz} % Работа с графикой
\usepackage{pgfplots}
\usepackage{pgfplotstable}

\usepackage[nottoc,numbib]{tocbibind}

\usepackage{titlesec}

\usepackage[titletoc]{appendix}



\begin{document} 
	
\section{Обзор литературы}	

\subsection{Деградация}

Одним из основных недостатков водород-воздушных топливных элементов на протон-обменных мембранах, который ограничивает их массовое внедрение является их малое время работы до необходимого обслуживания и ремонта \cite{Garland2007}. 
Именно поэтому ресурс топливных элементов является одной из основных тем исследований \cite{Yonoff2019}, причём доля статей, посвящённых этой теме, с каждым годом только растёт, что видно на рисунке \ref{fig:research_trends}.  

\begin{figure} [h]
	\centering
	\includegraphics[width=0.7\linewidth]{images/research_trends}
	\caption[Распределение статей по различным темам]{На графике обозначено распределение по годам доли статей по темам от всех публикаций по теме ПОМТЭ.}
	\label{fig:research_trends}
\end{figure}

Любой блок топливных элементов состоит из множества комплектующих, каждый из которых имеет свой ресурс. 
При этом ограниченное время работы всего устройства обычно связано со снижением характеристик либо мембраны, либо каталитического слоя \cite{Jahnke2016}.

\subsubsection{Мембрана}


Процессы, которые приводят к нарушению функций мембраны можно разделить на две крупные категории \cite{Collier2006}.
Первая категория --- это механическая и термическая деградация. К механической относят возникновение надрывов, трещин и отверстий, а к термической --- необратимое пересыхание и термолиз.
Ко второй категории можно отнести химические и электрохимические механизмы.
Наиболее важный процесс этой категории --- это образование гидропероксильного или какого-либо другого свободного радикала, а также дальнейшее разложение мембраны при радикальной атаке.
Важно отметить, что процессы двух категорий плотно связаны друг с другом. 
Так, например, химическая деградация мембраны может являться причиной сниженной прочности и появление трещин, а значит и большей подверженности отказу из-за механических повреждений \cite{Xie2016}.

%\paragraph{Nafion\texttrademark}

Для разных мембран будут преобладать разные процессы деградации. Так, например, механические повреждения на армированных мембранах появляются в 2--3 раза медленнее, чем на неармированных \cite{Ramani2020}.
Для большей конкретики эта работа
% {\tiny \textcolor{blue}{или этот раздел}} 
посвящена процессам разрушения перфторированных сульфополимерных мембран, таких как Nafion\texttrademark. Химическая структура этого полимера приведена на схеме \ref{nafion_structure}.

\def\vert{2.5 em}
\definesubmol{sidechain}{(-[6,0.75]CF_2 -[,0.75] CF(-[2,0.75]CF_3) -[,0.75] O -[,0.75] CF_2-[,0.75]CF_2 -[,0.75] SO_3 H)}
\definesubmol{leftchain}{(-[4,0.75]CF_2(-[4,0.5]))}

\begin{scheme} % Струтура Nafion
	\centering
	\vspace{2 em}
	\schemestart
	\chemfig{-[@{op,1}]CF_2-CF(-[6,0.75]O!{sidechain})-[,0.75]-{{(CF_2CF_2)}_m}-#(0.1em)-[@{cl,0}]}
	\polymerdelim[height = 1 em, depth = 2 em, open xshift = -1 em, close xshift = -1.5 em, indice = \!\!n]{op}{cl}
	\schemestop 
	\caption{Основная химическая структура Nafion\texttrademark.}
	\label{nafion_structure}
\end{scheme}
Некоторые техники, используемые для изготовления мембранно-электродного блока, в особенности, горячее прессование, могут оказывать большое влияние на 
%механическую и термическую стабильность 
работу и ресурс мембраны в готовом изделии \cite{Yazdanpour2012}. 
При этом в процессе функционирования готового топливного элемента температуры могут изменять механические характеристики мембраны \cite{Bauer2005}, но при этом лишь косвенно влиять на её деградацию. 
Так, они не способны привести к термическому разложению мембраны, так как для этого нужны температуры более 300 \celsius \cite{Yamaguchi2021}. 
Именно поэтому если говорить о долговечности, то для топливных элементов с перфторсульфополимерными мембранами важнейшую роль в их отказе будут играть механическая и химическая виды деградации \cite{Nguyen2021}.

Начнём рассмотрение разрушения мембраны с химических механизмов. Для проведения реакций, отвечающих за этот процесс, без необходимости сборки и испытания целого топливного элемента нередко применяют \cite{Tsuneda2020} реагент Фентона: раствор пероксида водорода $H_2O_2$ с ионами железа. 
Важно отметить ограниченную применимость выводов о стабильности мембраны, сделанных в результате такого моделирования.
Так, например, скорость разложения мембраны сильно зависит от давления \cite{Kusoglu2014}, а в топливном элементе мембрана обычно сжата между биполярными пластинами, поэтому результаты, полученные на свободной мембране, должны анализироваться с учётом этого и многих других эффектов.

%\paragraph{Химия}
Существует два варианта того, как может пойти процесс химического разложения мембраны: это распад основной гидрофобной цепи или отщепление боковых гидрофильных ответвлений. 
Механизмы отделения боковой цепи представлены на схеме \ref{sidechain_scisson}. 
\begin{scheme} % Отсоединение боковой цепи
	{\scriptsize   
		\schemestart
		\chemfig{(O((-[6,0.75]CF_2 -[,0.75] CF(-[2,0.75]CF_3) -[,0.75] O -[,0.75] CF_2-[,0.75]CF_2 -[,0.75] SO_3 H))(-[2,0.75]CF!{leftchain}-[,0.75]CF_2-[,0.75]))} \arrow	
		\chemfig{(O((=[6,0.75]CF -[,0.75] CF(-[2,0.75]CF_3) -[,0.75] O -[,0.75] CF_2-[,0.75]CF_2 -[,0.75] SO_3 H))(-[2,0.75,,,white]CF_2!{leftchain}-[,0.75]CF_2-[,0.75]))}
		\schemestop 
		\vspace{\vert}
		
		\schemestart
		\chemfig{(O((-[6,0.75]CF_2 -[,0.75] CF(-[2,0.75]CF_3) -[,0.75] O -[,0.75] CF_2-[,0.75]CF_2 -[,0.75] SO_3 H))(-[2,0.75]CF!{leftchain}-[,0.75]CF_2-[,0.75]))} \arrow
		\chemfig{(O((-[6,0.75,,,white]CF_3 -[,0.75] CF(-[2,0.75]CF_3) -[,0.75] O -[,0.75] CF_2-[,0.75]CF_2 -[,0.75] SO_3 H))(=[2,0.75]CF!{leftchain}-[,0.75]CF_2-[,0.75]))} 
		\schemestop
		
		\caption{Два варианта отщепления боковой цепи.}
		\label{sidechain_scisson}
	}
\end{scheme}
Существует два основных пути протекания этого процесса, которые отличаются местом разрыва эфирной связи, связывающей основную цепь с боковой.
Важно отметить, что первый из них, с двойной связью в отщеплённом остатке, случается гораздо чаще \cite{Yamaguchi2021}. 
После отсоединения гидрофильный остаток подвержен дальнейшему распаду, в основном путём диссоциации внутренней эфирной связи по схеме \ref{sidechain_further}.
\begin{scheme} % Распад боковой цепи
	{\small
		\schemestart
		\chemfig{(O((=[6,0.75]CF -[,0.75] CF(-[2,0.75]CF_3) -[,0.75] O -[,0.75] CF_2-[,0.75]CF_2 -[,0.75] SO_3 H)))} \arrow	
		\chemfig{(O((=[6,0.75]CF -[,0.75] CF_3(-[2,0.75]CF_3) -[,0.75,,,white] O =[,0.75] CF-[,0.75]CF_2 -[,0.75] SO_3 H)))}
		\schemestop 
		\caption{Дальнейший распад боковой цепи.}
		\label{sidechain_further}
	}
\end{scheme}

При нормальной работе топливного элемента следы отщепления вторичных цепочек обычно обнаруживаются только через очень продолжительное время \cite{Xie2016}, так что среди причин деградации, начинающейся сразу после запуска, преобладает распад основной цепи. 
Предполагается \cite{Kurniawan2013}, что за этот процесс отвечают радикальные атаки ионов $\, \cdot OH$, образование которых происходит схеме \ref{OH}.
\begin{scheme} %Возникновение ОН
	\centering
	$H_2 \rightarrow 2 H\cdot$ \\
	$H \cdot + O_2\cdot \rightarrow \cdot OOH$ \\
	$H \cdot + \cdot OOH \rightarrow H_2O_2$ \\
	$H_2O_2 + M^{2+} \rightarrow M^{3+}+ \cdot OH + OH^-$
	\caption{Возникновение ионов $\, \cdot OH$}
	\label{OH}
\end{scheme}

\noindent Рассмотрим эту схему подробнее. 
Для простоты будем считать, что все эти реакции протекают на анодной стороне мембраны, в дальнейшем будет показано, что это условие не обязательно.
Реакция диссоциации молекулы водорода на протоны происходит на платине в катализаторе и является стандартной реакцией в процессе работы топливного элемента.
После образования протонов они реагируют с молекулой кислорода, которая может продиффундировать сквозь мембрану или же поступить вместе с недостаточно очищенным водородом, и образуют молекулу пероксида водорода.
После этого на примесном ионе металла со степенями окисления 2+ и 3+, например железа или меди, который выступает в качестве катализатора, пероксид водорода распадается на ионы $\cdot OH$ и $OH^-$.
Для протекания этих же реакций на катоде мембранно-электродного блока так же необходимо наличие обоих газов и загрязнения в виде иона металла, однако в этом случае кислород подаётся при работе топливного элемента, а водород поступает в результате кроссовера. 
Можно предположить, что ионы металла со степенями окисления 2+ и 3+, необходимые для данной реакции, попадают на мембрану топливного элемента в том числе и из прижимных пластин, так как изменение их материала с нержавеющей стали на алюминиевый сплав позволяет значительно снизить темп деградации мембраны \cite{Pozio2003}.

После образования ионов $\cdot OH$ может начаться само химическое разложение основной цепи. 
Этапы этого процесса \cite{Chen2009} представлены на схеме \ref{unzip}.
В результате первых четырёх реакций происходит отщепление концевого элемента полимерной цепи, а так же первого блока $CF_2$. 
После этого эти реакции повторяются и продолжают протекать до тех пор, пока полимерная цепь не дойдёт до бокового ответвления. 
При наступлении такой ситуации может произойти гидролиз связи углерода с кислородом, боковая цепь отсоединится, а на конце основной образуется такая же концевая группа, которая была в самом начале этой схемы, после чего весь процесс может начаться заново. 
Отделившаяся боковая цепь может распасться по механизму, похожему на приведённый на схеме \ref{sidechain_further}, однако с другой концевой группой.

Существуют свидетельства в пользу того, что механизм образования ионов $\cdot OH$ из пероксида водорода $H_2O_2$ по схеме \ref{OH} и дальнейшей радикальной атаки на мембрану является доминирующим только при предельно низких токах \cite{Mittal2007}, а с ростом нагрузки его роль снижается.
Впрочем, по другим данным именно выдерживание МЭБа при напряжении открытой цепи разрушает мембрану больше других режимов работы \cite{Ohma2008}, особенно при подаче газов малой влажности \cite{Chen2009}.
Более того, сравнение изменений мембран в растворах, содержащем только ионы железа или их же с добавлением пероксида водорода $H_2O_2$ показывает, что, несмотря на различия в морфологии, разницы в химической деградации в этих растворах почти нет \cite{Kundu2008}.
По результатам изучения разрушения мембраны в составе МЭБа, а не в растворах, можно сделать вывод, что для разложения мембраны нужны оба подаваемых газа и углеродно-платиновый электрод, а при отсутствии хотя бы одного из этих условий деградация мембраны почти останавливается \cite{Ghassemzadeh2010}. 
Этот факт объясняется тем, что первый из этапов в схеме \ref{OH} возникновения ионов $\cdot OH$ диссоциации молекулы водорода на протоны не протекает без платинового катализатора.

\begin{scheme} % Основная цепь
	{\scriptsize   
	%\tiny
%	\setchemfig{bond offset=1em}
	\schemestart
	\chemfig{(O!{sidechain}(-[2,0.75]CF!{leftchain}-[,0.5]-{{(CF_2CF_2)}_m}--[,0.5]COOH))}
	\arrow{->[+ \charge{180=\.}{OH}][$-CO_2 \; -H_2O$]}[0,1.5]
%	\schemestop 
%	\vspace{\vert}
%	
%	\schemestart
	\chemfig{(O!{sidechain}(-[2,0.75]CF!{leftchain}-[,0.75]-#(,0em){{(CF_2CF_2)}_{m-1}}-#(0.1em)-\charge{0=\.}{CF_2CF_2}))}
	\arrow{->[+ \charge{180=\.}{OH}][]}
	\schemestop
	\vspace{\vert}
	
	\schemestart
	\chemfig{(O!{sidechain}(-[2,0.75]CF!{leftchain}-[,0.75]-#(,0em){{(CF_2CF_2)}_{m-1}}-#(0em)-[,0.75]CF_2CF_2-OH))}
	\arrow{->[][-HF]}
%	\schemestop
%	\vspace{\vert}
%	
%	\schemestart
	\chemfig{(O!{sidechain}(-[2,0.75]CF!{leftchain}-[,0.75]-#(,0em){{(CF_2CF_2)}_{m-1}}-#(0em)-[,0.75]CF_2C(-[6,0.75,3]F)=[,0.75]O))}
	\arrow{->[$+H_2O$][-HF]}
	\schemestop
	\vspace{\vert}
	
	\schemestart
	\chemfig{(O!{sidechain}(-[2,0.75]CF!{leftchain}-[,0.75]-#(,0em){{(CF_2CF_2)}_{m-1}}-#(0em)-[,0.75]CF_2-[,0.75]COOH))}
	\arrow{->[+(4m-1)  \charge{180=\.}{OH}][$-2m \, CO_2 \; \; -(4m-2) HF \; \; -H_2O$]}[,3.5]
	\schemestop 
	\vspace{\vert}
	
	\schemestart
	\chemfig{(O!{sidechain}(-[2,0.75]\charge{0=\.}{CF}!{leftchain}))}
	\arrow{->[+  \charge{180=\.}{OH}][-HF]}
	\chemfig{(O!{sidechain}(-[2,0.75]C!{leftchain}=[,0.75]O))}
	\arrow{->[+$H_2O$][]}
	\schemestop
	\vspace{\vert}
	
	\schemestart
	\chemfig{(OH(-[2,0.75]C!{leftchain}=[,0.75]O))}
	+
	\chemfig{OH!{sidechain}}
	%\chemfig{CF_2(-[2,0.75]OH) -[,0.75] CF(-[2,0.75]CF_3) -[,0.75] O -[,0.75] CF_2CF_2 -[,0.75] SO_3 H}
	\arrow{->[$+H_2O$][-2HF]}
	\schemestop 
	\vspace{\vert}
	
	\schemestart
	\chemfig{(O!{sidechain}(-[2,0.75]CF!{leftchain}-[,0.5]-{{(CF_2CF_2)}_n}--[,0.5]COOH))}
	+
	\chemfig{HOOC -[,0.75] CF(-[2,0.75]CF_3) -[,0.75] O -[,0.75] CF_2CF_2 -[,0.75] SO_3 H}
	\schemestop 
	\caption{Процесс расцепления основной цепи.}
	\label{unzip}
}
\end{scheme}

%\paragraph{Механика}
Переходя к категории механических разрушений мембраны, необходимо сказать о каких именно повреждениях будет идти речь. 
Прежде всего это отверстия, а так же трещины и надрывы \cite{Kusoglu2011}, в том числе сквозные. 
В дополнение к ним нужно отметить пузыри, надувающиеся и лопающиеся вследствие химической деградации \cite{Fernandes2009}.
В экспериментах с ускоренными стресс-тестами было получено два различных вида пузырей, отличающихся в первую очередь размером: он может быть сравним с толщиной мембраны или быть много меньше. 
Пузыри первого вида являются одной из причин роста кроссовера, так как они могут приводить к расслаивании мембраны и возникновению отверстий при схлопывании, а второго --- почти не влияют на газопроницаемость, так как они лишь меняют морфологию поверхности полимера, делая её пористой, и почти не затрагивают центральные слои \cite{Kundu2008}.

Трещины в мембране, как и каталитическом слое, которые хорошо заметны при изучении сухого МЭБ, могут полностью закрыться после увлажнения благодаря поглощению воды и набуханию \cite{Ramani2020}.

\subsubsection{Каталитический слой}

Описывая деградацию каталитического слоя, нельзя не сказать про миграцию агломератов платины в мембрану.
%, которая происходит при выдерживании мембранно-электродного блока при напряжении открытой цепи
Помимо того, что этот процесс вредит катализатору, так как частицы платины перестают быть доступными для газов и электронного тока, он  нарушает целостность мембраны, а так же способствует её дальнейшему разрушению \cite{Ohma2008}.
В состав каталитического слоя, помимо других элементов, входит иономер, который по своим способам деградация во многом похожа на мембрану \cite{Okonkwo2021_I}. 
Однако катализатор подвержен и множеству других процессов, снижающих его работоспособность. 

Среди основных механизмов деградации можно назвать отравление, вымывание, оствальдовское созревание и миграцию частиц платины \cite{Okonkwo2021_Pt}, однако помимо них важны и многие другие процессы \cite{Nguyen2021}. 
Важно отметить, что отдельные аспекты этих механизмов неплохо изучены  \cite{Garcia2008, Prokop2020}, однако причину снижения характеристик топливного элемента в каждом конкретном случае удаётся понять далеко не всегда.


\subsection{Способы изучения}

Существует множество экспериментальных методов, позволяющих отслеживать изменения в мембранно-электродных блоках топливных элементов, происходящих вследствие деградации. 
Прежде всего это контроль снижения характеристик при проведении ускоренных стресс-тестов.
Такой подход позволяет предсказывать ресурс, а так же отслеживать влияние различных изменений в материалах или способах изготовления топливного элемента.
Впрочем, этот способ не лишён недостатков, важнейшим из которых является тот факт что далеко не всегда по снижению характеристик можно понять, какой именно процесс является лимитирующим.
Таким образом, несмотря на возможность делать прогнозы о сроке службы, механизм деградации остаётся непонятным.

Описание причин падения характеристик почти невозможно без изучения мембранно-электродных блоков до и после ускоренных стресс-тестов. 
Для анализа изменений структуры применяются разнообразные техники, каждая из которых обладает рядом достоинств и недостатков. 
Так, например, использование оптических и сканирующих электронных микроскопов весьма популярно по причине простоты и доступности этих методов, однако ограничения в виде сравнительно низкого разрешения и возможности изучения только приповерхностных слоёв вынуждают искать другие подходы. 
Просвечивающая электронная микроскопия позволяет изучать образцы по всему объёму, однако резко ограничивает его максимальную толщину.

{\tiny \textcolor{blue}{broad-band dielectric
	spectroscopy,32 EPR,8,9,30,33-40 FT-IR,15,20,24,27,33,41-44 Raman,33
	UV-visible,33 liquid24,45,46 and solid-state20,24,42,47 NMR, mass
	spectroscopy,45,46 XPS,10,26,41,48,49 wide-angle and small-angle
	X-ray diffraction,10,26 TGA,27,28,50 TEM,51 and SEM,24 \cite{Ghassemzadeh2010} + ion release}}

\subsubsection{Синхротрон}

Основной темой этой работы является использование дифракции рентгеновского синхротронного излучения, так как такой подход обладает целым рядом преимуществ. 
Данные о широкоугловом рентгеновском рассеянии позволяют с хорошей точностью смоделировать структуру платины в каталитическом слое, описав размер и кристалличность её частиц \cite{Sasaki2016}.
Малоугловое рассеяние может использоваться для получения информации об иономере и макроструктуре каталитического слоя, а так же о характеристиках мембраны \cite{Schmidt2008}.
Микро- и нано-фокусная дифракция позволяют исследовать распределение, например, размера частиц платины в каталитическом слое от расстояния от мембраны, получая более полную информацию о состоянии мембранно-электродного блока. 
Благодаря высокой яркости синхротронного источника есть возможность изучения топливного элемента во время работы при использовании специальной ячейки \cite{Martens2019_cell}, что недоступно для абсолютного большинства других методов изучения.

Значительная доля опубликованных на данный момент исследований мембранно-электродных блоков в рентгеновском диапазоне посвящены изучению распределения и транспорта воды, причём для исследований используются как лабораторные рентгеновские микроскопы \cite{Lee2013, Eller2014}, так и синхротроны \cite{Sasabe2011, Deevanhxay2011}. 
Другая часть таких работ посвящена различным аспектам одного из важнейших процессов, обеспечивающих работу топливного элемента --- окислению наночастиц платины. 
Темами различных исследований становились этапы окисления частиц на углеродной подложке \cite{Imai2009}, в различных типах электрохимических ячеек \cite{Sasaki2016} и в составе МЭБов \cite{Martens2019_MEA}.
Немаловажно выделить исследование неоднородности этого процесса по поверхности каталитического слоя и, как следствие, различной скорости деградации топливного элемента в разных его частях \cite{Martens2021}.

\subsection{ТОТЭ}

Гораздо большую популярность синхротроны снискали с другими топливными элементами --- твёрдооксидными.
Это легко объяснить гораздо более выраженной кристаллической структурой, изменения которой легко заметить с помощью рентгеновской дифракции.
Это позволяет отслеживать напряжения кристаллической решетки \cite{Villanova2010}, определять микроструктуру \cite{Shearing2011} и даже строить трёхмерные модели этого типа топливных элементов \cite{Izzo2008}. 
Также можно отметить некоторые работы, посвящённые изменениям в структуре ТОТЭ при длительной работе \cite{Villanova2019}.


\section{Презентация}

Для проведения исследования было изготовлено два комплекта образцов. 
Каждый комплект включал в себя три образца, из которых один не использовался после изготовления, второй проходил только процедуру активации, а третий помимо неё подвергался деградационным испытаниям.
Все изучавшиеся образцы изготавливались методом декалькирования --- после приготовления каталитических чернил они наносились на подложку, которая после высыхания припресовывалась к мембране, переводя каталитический слой на ней.
За основу была взята стандартная технология производства в Инэнерджи, которая, не позволяет достичь высоких показателей сохраняемости мембранно-электродных блоков в процессе их работы.
Для решения этой проблемы было необходимо провести ряд экспериментов, для чего пришлось адаптировать технологию изготовления для производства единичных образцов.
На основе литературных данных и собственных экспериментов было принято решение отказаться от нанесения каталитических чернил на подложку в процессе трафаретной печати и перейти к напылению аэрографом.
Этот способ позволял использовать значительно меньшие объёмы чернил, изготавливая только необходимое для проверки технологии число образцов.   
Так как процесс производства мембранно-электродных блоков зависит от множества факторов, сменить технологию, изменив только способ нанесения не получилось.
В частности, с целью достижения более оптимального перевода каталитического слоя на мембрану пришлось заменить материал подложки с армированного тефлона на лавсан.
Важно отметить, что эти изменения привели к различным загрузкам платины в разных образцах, поэтому сравнивать абсолютные значения полученных характеристик может быть некорректно, в отличие от отслеживания их изменений в процессе работы.

Во многих работах для тестирования сохраняемости обычно МЭБ либо выдерживают в потенциостатическом режиме, либо подвергают импульсной нагрузке, регистрируя поляризационные кривые и другие парметры до после ускоренного стресс-теста.
При проведении этого эксперимента проводилось множество циклических ВАХ - развёрток по напряжению в две стороны с записью силы тока при каждом напряжении.
В отличие от других популярных методов проведения деградационных испытаний, этот метод позволяет следить за изменением характеристик в динамике.
Важно отметить, что на поляризационных кривых был виден гистерезис, то есть величина тока при каждом напряжении зависела от направления сканирования.
Это явление будет более подробно описано ниже.

Так как для каждого образца было записано более 5000 циклов, было крайне сложно выявить какие-либо закономерности в изменении характеристик непосредственно из самих данных, поэтому использовался следующий способ их анализа. 
Для нескольких напряжений с шагом 0.1 В было построено по графику. 
Каждая точка на каждом графике показывает силу тока при фиксированном напряжении. 
Чёрный и красный цвета отвечают за силу тока при сканировании вверх и вниз соответственно. 
На графиках отображены как сырые данные, так и апроксимационные кривые по каждому направлению и по всем точкам сразу. 
Эти графики позволяют увидеть, что разница между двумя цветами, отвечающая за величину гистерезиса меняется с течением времени. 
При напряжении 0.6 В гистерезис очень мал, что объясняется наличием точки пересечения ВАХ на этом напряжении. 
Видно, что при более высоком напряжении сила тока при сканировании вверх выше силы тока при сканировании вниз, а при более низком – наоборот.

В целях безопасности эксперимент прерывался на ночь, поэтому мы видим скачки внутри дня, которые могут быть даже больше, чем между ними.
Это явление описано в литературе и называется обратимой деградацией, то есть той, которая исчезает при прекращении работы и повторном запуске.
Она связана с локальными изменениями, то есть затоплением или пересушением мембраны или каталитических слоёв, в отличие от необратимой, связанной со структурной перестройкой внутри МЭБ.
На графиках для 0.3 и 0.4 В наблюдается аномальный подъем, предположительно связанный с выравниванием ВАХ в области высоких токов и установлении водного баланса. 
Однако эти напряжения почти не используются при работе из-за более низкой мощности и больших проблем с затоплением МЭБ.
Видно, что в целом поведение образца сильно отличается, поэтому нам и важно дальше сравнивать МЭБы, которые были сделаны по разным технологиям.

Важнейшим результатом проведенных экспериментов является возможность разделить обратимую и необратимую деградацию. 
Этот результат крайне важен, так как на примере второго образца видно, что обратимая деградация может оказывать основное влияние на характеристики топливного элемента в процессе его работы. 
На основании этого становится особенно важно главное достоинство используемого метода позволяющего отслеживать характеристики в динамике, в отличие от других подходов к ресурсному тестированию МЭБ не имеющих такой возможности.
Так же можно отметить, что точка пересечения ВАХ с течением времени смещается в сторону больших плотностей тока, а гистерезис уменьшается, что свидетельствует о наступлении водного баланса.



\printbibliography[heading=bibintoc,title={Список литературы}]

\end{document}